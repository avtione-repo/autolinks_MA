%!TEX encoding = UTF-8 Unicode
\documentclass[
    fontsize=12pt,
    headings=small,
    parskip=half,           % Ersetzt manuelles setzten von parskip/parindent.
    bibliography=totoc,
    numbers=noenddot,       % Entfernt den letzten Punkt der Kapitelnummern.
    open=any,               % Kapitel kann auf jeder Seite beginnen.
%   final                   % Entfernt alle todonotes und den Entwurfstempel.
    ]{scrreprt}

% ===================================Praeambel==================================

% Kodierung, Sprache, Patches {{{
\usepackage[T1]{fontenc}    % Ausgabekodierung; ermoeglicht Akzente und Umlaute
                            %  sowie korrekte Silbentrennung.
\usepackage[utf8]{inputenc} % Erlaub die direkte Eingabe spezieller Zeichen.
                            %  Utf8 muss die Eingabekodierung des Editors sein.
% \usepackage[ngerman]{babel} % Deutsche Sprachanpassungen (z.B. Ueberschriften).
\usepackage{microtype}      % Optimale Randausrichtung und Skalierung.
\usepackage[
    autostyle,
    ]{csquotes}             % Korrekte Anfuehrungszeichen in der Literaturliste.
\usepackage{fixltx2e}       % Patches fuer LaTeX2e.
\usepackage{scrhack}        % Verhindert Warnungen mit aelteren Paketen.
\usepackage[
  newcommands
]{ragged2e}                 % Verbesserte \ragged...Befehle
\PassOptionsToPackage{
  hyphens
}{url}                      % Sorgt für URL-Umbrueche in Fusszeilen u. Literatur
% }}}

% Schriftarten {{{
\usepackage{mathptmx}       % Times; modifies the default serif and math fonts
\usepackage[scaled=.92]{helvet}% modifies the sans serif font
\usepackage{courier}        % modifies the monospace font
% }}}

\usepackage{indentfirst}
\setlength{\parindent}{0.5cm}

% Biblatex {{{
\usepackage[
    style=alphabetic,
    backend=biber,
    %backref=true
    ]{biblatex}             % Biblatex mit alphabetischem Style und biber.
\bibliography{\jobname.bib}      % Dateiname der bib-Datei.
\DeclareFieldFormat*{title}{
    \mkbibemph{#1}}         % Make titles italics
% }}}

% Dokument- und Texteinstellungen {{{
\usepackage[
    a4paper,
    margin=2.54cm,
    marginparwidth=2.0cm,
    footskip=1.0cm
    ]{geometry}             % Ersetzt 'a4wide'.
\clubpenalty=10000          % Keine Einzelzeile am Beginn eines Paragraphen
                            %  (Schusterjungen).
\widowpenalty=10000         % Keine Einzelzeile am Ende eines Paragraphen
\displaywidowpenalty=10000  %  (Hurenkinder).
\usepackage{floatrow}       % Zentriert alle Floats.
\usepackage{ifdraft}        % Ermoeglicht \ifoptionfinal{true}{false}
\pagestyle{plain}           % keine Kopfzeilen
% \sloppy                    % großzügige Formatierungsweise
\deffootnote{1em}{1em}{
  \thefootnotemark.\ }      % Verbessert Layout mehrzeiliger Fußnoten

\makeatletter
\AtBeginDocument{%
    \hypersetup{%
        pdftitle = {\@title},
        pdfauthor  = \@author,
    }
}
\makeatother
% }}}

% Weitere Pakete {{{
\usepackage{graphicx}       % Einfuegen von Graphiken.
\usepackage{tabu}           % Einfuegen von Tabellen.
\usepackage{multirow}       % Tabellenzeilen zusammenfassen.
\usepackage{multicol}       % Tabellenspalten zusammenfassen.
\usepackage{booktabs}       % Schönere Tabellen (\toprule\midrule\bottomrule).
\usepackage[nocut]{thmbox}  % Theorembox bspw. fuer Angreifermodell.
\usepackage{amsmath}        % Erweiterte Handhabung mathematischer Formeln.
\usepackage{amssymb}        % Erweiterte mathematische Symbole.
\usepackage{rotating}
\usepackage[
    printonlyused
    ]{acronym}              % Abkuerzungsverzeichnis.
\usepackage[
    colorinlistoftodos,
    textsize=tiny,          % Notizen und TODOs - mit der todonotes.sty von
    \ifoptionfinal{disable}{}%  Benjamin Kellermann ist das Package "changebar"
    ]{todonotes}            %  bereits integriert.
\usepackage[
    breaklinks,
    hidelinks,
    pdfdisplaydoctitle,
    pdfpagemode = {UseOutlines},
    pdfpagelabels,
    ]{hyperref}             % Sprungmarken im PDF. Laed das URL Paket.
    \urlstyle{rm}           % Entfernt die Formattierung von URLs.
%\usepackage{breakurl}
%\def\UrlBreaks{\do\/\do-}
\usepackage{listings}       % Spezielle Umgebung für...
    \lstset{                %  ...Quelltextformatierung.
        language=C,
        breaklines=true,
        breakatwhitespace=true,
        frame=L,
        captionpos=b,
        xleftmargin=6ex,
        tabsize=4,
        numbers=left,
        numberstyle=\ttfamily\footnotesize,
        basicstyle=\ttfamily\footnotesize,
        keywordstyle=\bfseries\color{green!50!black},
        commentstyle=\itshape\color{magenta!90!black},
        identifierstyle=\ttfamily,
        stringstyle=\color{orange!90!black},
        showstringspaces=false,
        }
\usepackage{filecontents}   % Direktes Einfuegen von Dateiinhalt. Wird hier fuer
                            %  die Verwendung einer .bib-Datei in dieser .tex-
                            %  Datei benoetig.
% }}}

% =================================Bibliographie================================

\begin{filecontents}{\jobname.bib}
% ------------------------------------Buecher-----------------------------------

@book{Beut2009,
  %required: author,title,publisher,year
  %optional: volume,number,series,address,edition,month,isbn,note
  author={Beutelspacher, Albrecht},
  title={{Kryptologie: Eine Einführung in die Wissenschaft vom Verschlüsseln,
    Verbergen und Verheimlichen}},
  publisher={Vieweg + Teubner},
  year={2009},
  address={Wiesbaden},
  edition={9. akt. Auflage},
  % isbn={9783834896063}, % ISBN-Angabe bitte möglichst nicht verwenden
}

@book{Posp2012,
  %required: author,title,publisher,year
  %optional: volume,number,series,address,edition,month,isbn,note
  author={Pospiech, Ulrike},
  title={{Wie schreibt man wissenschaftliche Arbeiten? -- Alles Wichtige von der
    Planung bis zum fertigen Text}},
  publisher={Dudenverlag},
  year={2002},
  address={Mannheim/Zürich},
  % isbn={9783411747115}, % ISBN-Angabe bitte möglichst nicht verwenden
}

@book{Schl2013,
  %required: author,title,publisher,year
  %optional: volume,number,series,address,edition,month,isbn,note
  author={Schlosser, Joachim},
  title={{Wissenschaftliche Arbeiten schreiben mit LaTeX -- Leitfaden für
    Einsteiger}},
  publisher={mitp-Verlag},
  year={2013},
  address={Heidelberg},
  edition={5. Auflage},
  % isbn={9783826694868}, % ISBN-Angabe bitte möglichst nicht verwenden
}

@book{ScWe2007,
  %required: author,title,publisher,year
  %optional: volume,number,series,address,edition,month,isbn,note
  author={Schneider, Uwe},
  title={{Taschenbuch der Informatik}},
  publisher={Carl-Hanser-Verlag},
  year={2007},
  address={Leipzig},
  edition={6. Auflage},
  % isbn={9783446407541}, % ISBN-Angabe bitte möglichst nicht verwenden
}

% ------------------Artikel einer Zeitung oder eines Magazins-------------------

@article{Kili2006,
  %required: author,title,journal,year
  %optional: volume,number,pages,month,note,doi/url+urldate
  author={Kilian, Detlef},
  title={{Einführung in Informationssicherheitsmanagementsysteme (I) -
    Begriffsbestimmung und Standards}},
  journal={Datenschutz und Datensicherheit DuD},
  year={2006},
  volume={30},
  number={10},
  pages={651-654},
}

@article{Lamp1981,
  %required: author,title,journal,year
  %optional: volume,number,pages,month,note,doi/url+urldate
  author={Lamport, Leslie},
  title={{Password authentication with insecure communication}},
  journal={Communications of the ACM},
  year={1981},
  volume={24},
  number={11},
  pages={770-772},
}

@article{ThKZ2002,
  %required: author,title,journal,year
  %optional: volume,number,pages,month,note,doi/url+urldate
  author={Thalheim, Lisa and Krissler, Jan and Ziegler, Peter-Michael},
  title={{Körperkontrolle -- Biometrische Zugangssicherungen auf die Probe
    gestellt}},
  journal={ct},
  year={2002},
  number={11},
  pages={114--123},
}

% ----------------------------- Konferenzbeitraege -----------------------------

@inproceedings{HSFN2009,
  %required: author,title,booktitle,year,editor,publisher
  %optional: series,volume,number,pages,address,month,organization,note,
  %  doi/url+urldate
  author={Herrmann, Dominik and Scheuer, Florian and Feustel, Philipp and Nowey,
    Thomas and Federrath, Hannes},
  title={{A Privacy-Preserving Platform for User-Centric Quantitative
    Benchmarking}},
  booktitle={Proceedings of the 6th International Conference on Trust, Privacy
  and Security in Digital Business},
  year={2009},
  editor={Fischer{-}Hübner, Simone and Lambrinoudakis, Costas and Pernul,
    Günther},
  volume={5695},
  series={Lecture Notes in Computer Science},
  pages={32-41},
  address={Berlin},
  publisher={Springer-Verlag},
}

@inproceedings{InBr2009,
  %required: author,title,booktitle,year,editor,publisher
  %optional: series,volume,number,pages,address,month,organization,note,
  %  doi/url+urldate
  author={Innerhofer-Oberperfler, Frank and Breu, Ruth},
  title={{An empirically derived loss taxonomy based on publicly known security incidents}},
  booktitle={Proc. of International Conference on Availability, Reliability and Security (ARES'09)},
  year={2009},
  pages={66-73},
  organisation={IEEE},
  address={Fukuoaka, Japan},
  month={mar},
}

@inproceedings{WWPK2010,
  %required: author,title,booktitle,year,editor,publisher
  %optional: series,volume,number,pages,address,month,organization,note,
  %  doi/url+urldate
  author={Westermann, Benedikt and Wendolsky, Rolf and Pimenidis, Lexi and
    Kesdogan, Dogan},
  title={{Cryptographic Protocol Analysis of AN.ON}},
  booktitle={Financial Cryptography and Data Security: 14th International
    Conference, FC 2010},
  year={2010},
  editor={Sion, Radu},
  series={Lecture Notes in Computer Science},
  volume={6052},
  pages={114-128},
  address={Canary Islands, Spain},
  month={jan},
  publisher={Springer Science \& Business Media},
  url={http://dx.doi.org/10.1007/978-3-642-14577-3_11},
  urldate={2016-08-02},
}

% ==================================Onlinequellen==================================

@online{Bier2009,
  %required: author,title,year,url+urldate
  %optional: version,organisation,month,note
  author={Bier, Christoph},
  title={{typokurz -- Einige wichtige typographische Regeln)}},
  year={2009},
  month={may},
  url={https://zvisionwelt.files.wordpress.com/2012/01/typokurz.pdf},
  % https://zvisionwelt.wordpress.com/downloads/
  urldate={2016-07-13},
}

@online{CCC2009,
  %required: author,title,year,url+urldate
  %optional: version,organisation,month,note
  author={Kurz, Constanze and Rieger, Frank},
  title={{Chaos Computer Club veröffentlicht Stellungnahme zur
    Vorratsdatenspeicherung}},
  year={2009},
  month={jul},
  url={http://www.ccc.de/updates/2009/vds-gutachten},
  urldate={2014-12-05},
}

@online{faui2,
  %required: author,title,year,url+urldate
  %optional: version,organisation,month,note
  author={Friedrich-Alexander Universität Erlangen-Nürnberg},
  title={{Beurteilung von wissenschaftlichen Arbeiten am Lehrstuhl für
    Informatik 2 (Programmiersysteme)}},
  year={2012},
  month={mar},
  url={https://www2.informatik.uni-erlangen.de/teaching/thesis/review.html},
  urldate={2014-12-05},
}

@online{Heise2011,
  %required: author,title,year,url+urldate
  %optional: version,organisation,month,note
  author={Heise Security News},
  title={{US-Professor wirft Sony Mitschuld am PSN-Hack vor}},
  year={2011},
  month={may},
  url={http://www.heise.de/-1238676},
  urldate={2014-12-05},
}

@online{textwahrnehmung,
  %required: author,title,year,url+urldate
  %optional: version,organisation,month,note
  author={Spiegel Online},
  title={{Textwahrnehmung - Simple Sprache wirkt intelligenter}},
  year={2005},
  month={nov},
  url={http://www.spiegel.de/wissenschaft/mensch/0,1518,382730,00.html},
  urldate={2014-12-05},
}

@online{Wiki,
  %required: author,title,year,url+urldate
  %optional: version,organisation,month,note
  author={Wikipedia},
  title={{Enigma (Maschine)}},
  year={2011},
  month={apr},
  url={http://de.wikipedia.org/w/index.php?title=Enigma_(Maschine)&oldid=88241310},
  urldate={2014-12-05},
}

% ==================================Weitere==================================

%@incollection{<key>,
%  %required: author,title,booktitle,year,editor,publisher
%  %optional: series,volume,number,type,chapter,pages,address,edition,
%  %  month,isbn,note,doi/url+urldate
%}

%@patent{<key>,
%  %required: author,title,number,year
%  %optioal: holder,version,month,type,address,note,doi/url+urldate
%}

%@thesis{<key>,
%  %required: author,title,type,institution,year
%  %optional: month,pubstate,address,note,doi/url+urldate
%}

%@unpublished{<key>,
%  %required: author,title,year
%  %optional: month,pubstate,address,note,url+urldate
%}

@misc{Tolk2003,
  %required: author,title,year
  %optional: howpublished,version,organisation,month,address,note,url+urldate
  author={Tolksdorf, Robert},
  title={{Wie halte ich ein Referat und wie schreibe ich ein Papier}},
  year={2003},
  howpublished={Präsentation},
  address={FU Berlin},
}
\end{filecontents}

% ===================================Dokument===================================

\title{Master Thesis}
\author{Alvin Rindra Fazrie}
% \date{01.01.2015} % falls ein bestimmter Tag eingesetzt werden soll, einfach
                    %  diese Zeile aktivieren

\begin{document}

\begin{titlepage}
\includegraphics[width=6.8cm]{../pic/up-uhh-logo-u-2010-u-farbe-u-rgb.pdf}
  \setcounter{page}{-1}

	% Titelseite
	%\begin{figure}[h]
	%	\begin{minipage}[b]{62mm}
	%		\includegraphics[width=62mm]{../pic/up-uhh-logo-u-2010-u-farbe-u-rgb.pdf}
	%	\end{minipage}
	%	\hspace{4cm}
		%\begin{minipage}[b]{59mm}
		%	\includegraphics[width=59mm]{images/minlogo}
		%\end{minipage}
	%\end{figure}

	\vfill
\begin{center} 
		\noindent { \huge
			Master thesis \\
		}
		\vspace{14mm}
		% Titel
		\noindent \textbf{\huge
		  Autolinks: Information Management on Hypergraph of Semantic Triples 
		}
		\vspace{60mm}	
	\end{center}
	
	\vfill
	
	\noindent{\textbf{Alvin Rindra Fazrie}} \\
	\noindent \rule{\textwidth}{0.4mm} 
	\noindent{\textrm{4fazrie@informatik.uni-hamburg.de}} \\
	\noindent{\textrm{Intelligent Adaptive Systems Master program}} \\
	\noindent{\textrm{Matr.-Nr. 6641834}} \\
	\begin{tabbing}
	\hspace{8em} \=  \kill
	First Supervisor: \> Prof. Dr. Chris Biemann \\
	Second Supervisor: \> Steffen Remus MSc. \\
	\end{tabbing}
\end{titlepage}


\chapter*{Abstract}

 Autolinks 'automatic proactive researching’ is a tool that provides a quick researching platform based on a text or a sentence by visualizing the results together with their semantic relations. In this internet era, people could get information easily with search engines. They will give us a ton of hyperlinks clustered by multiple pages by entering a single query to the input, then we could select a specific link we think the most relevant. The process of learning takes a time sometimes. After the chosen web page rendered, we need to read through a page to get a specific information related to the query given and sometimes we still have to deal with a number of hyperlinks to get further information. Even worse, most of the website nowadays exploit the curiosity gap of the reader, providing just enough information and not enough to satisfy the reader’s curiosity, without clicking through another linked content. This clickbait phenomenon becomes so normal today and it makes our time to study longer. 
 
    	Autolinks optimizes these concerns and is intended to make the learning process faster and more efficient. Instead of reading papers, websites, and other resources to understand a specific term, this machine will do it for us. From a text or a sentence given by the user, it will read and learn from multiple resources and digests the core related information by visualizing the information in the most convenient way. The information is visualized by a force-directed graph, a graph which contains nodes for the information and edges for the semantic relation so that it will ease the reader to understand how pieces of information correlate each other.
    	
    	Autolinks is built with machine learning paradigm. Natural Language Processing (NLP) takes a responsibility to understand a given text and to comprehend which information from the sources have a relation to the given text and correlate each other. The reader could evaluate the results given and Autolinks will learn and correct the mistakes so that it could improve the precision and confidence in the next iteration. Bundled with this capability, Autolinks accelerates the process of researching and understanding during the study.  
    	
	With respect to the background and the purpose of Autolinks, we address some research questions in this master thesis, including the following: how can a user interface be devised that is non-intrusive, i.e. helping users solve their information needs faster instead of impeding them?; which semantic services, realized with NLP technologies, are the most useful?; how can we measure success, i.e. showing that Autolinks really live up to its premise?

\chapter*{Acknowledgement}

Here comes the acknowledgement...


\tableofcontents

\chapter{Introduction}

Here comes the Introduction...


\section{Motivation}

Here comes the motivation...


\section{Research question}

Here comes the the Research Questions...


\section{Contributions}

Here comes the contributions..


\chapter{Related Work}

Here comes Related Work...


\section{Information Management tools}

Here comes information management tools...


\section{Text annotation tools}

Here comes text annotation tools...



\chapter{Background Study}

Here comes Background Study...


\section{Language Technology}

Here comes contributions...


\section{Machine learning}

Here comes Machine Learning...


\section{Data Visualization}

Here comes Data Visualization...

\section{Hypergraph}

Here comes Hypergraph...

\section{Web technologies}

Here comes Web technologies...



\chapter{System Overview}

Here comes System Overview...

\section{Autolinks Introduction}

Here comes Autolinks Introduction...


\section{Components in Autolinks}

Here comes components in Autolinks...


\chapter{Data Extraction}

Here comes Data Extraction...

\section{Broker}

Here comes broker...

\section{Wiki Service}

Here comes Wikiservice...


\chapter{Information Management Visualization}

Here comes Information Management Viz...

\section{Concept and Visualization}

Here comes Concept and Visualization...

\section{Compound Nodes / Parent}

Here comes Compound Nodes / Parent...



\section{System Overview}

Here comes the System Overview...

\section{Data Extraction}

Here comes the Data Extraction...

Example of lists:

\begin{enumerate}
	\item Fachbücher, Standards,
	\item Wiss. Zeitschriftenartikel, Survey-Artikel,
	\item Konferenzbeiträge,
	\item Technical Reports, graue Literatur,
	\item Online-Material, Arbeitspapiere, Firmenmaterial, Ausarbeitungen.
\end{enumerate}

Im Internet können zur Feststellung der Qualität und Recherche von Publikationen

\begin{itemize}
	\item Google Scholar (\url{http://scholar.google.com}),
	\item Microsoft Academic Search (\href{http://academic.research.microsoft.com/?SearchDomain=2&SubDomain=2&entitytype=2}{http://academic.research.microsoft.com}) $\to$ computer science $\to$ security \& privacy,
	\item Computer Science Bibliography (\url{http://dblp.uni-trier.de/}) und die
	\item Scientific Literature Digital Library (\url{http://citeseer.nj.nec.com/})
\end{itemize}



\chapter{Evaluation}

Here comes the evaluation...

\section{Case Study}

Here comes the case study...

\section{User Experiment}

Here comes the user experiment...

\section{Evaluation details}

Here comes the evaluation details...

\begin{lstlisting}[float,caption={Example of algorithm},label={lst:ggt}]
int getGGTOf(int a, int b) {
    // requires ((a > 0) && (b > 0)); ensures return > 0;
    int h;
    while (b != 0) {
        h = b;
        b = a % b; // % is the modulo operator. This line is long enough to show how line breaks in lstlisting are handled.
        a = h;
    }
    return a;
}
\end{lstlisting}


\chapter{Future Work}

Here comes the future work...


\chapter{Conclusion}

Here comes the conclusion...

% =============================Literaturverzeichnis=============================
\begin{raggedright}         % Schaltet Blocksatz ab, erzeugt ein stimmigeres
                            %  Schriftbild im Literaturverzeichnis.
  \printbibliography        % Falls Biblatex verwendet wird.
  \label{sec:literaturverzeichnis}
\end{raggedright}



% ===========================Selbststaendigkeitserklaerung======================
\chapter*{Eidesstattliche Versicherung} % war: Selbständigkeitserklärung
\vspace{1cm}

\todo[noline]{Bitte verwenden Sie hier in jedem Fall die offizielle von der Prüfungsbehörde vorgegebene Formulierung der Selbständigkeitserklärung.}
%
Hiermit versichere ich an Eides statt, dass ich die vorliegende Arbeit selbstständig verfasst und keine anderen als die angegebenen Hilfsmittel – insbesondere keine im Quellenverzeichnis nicht benannten Internet-Quellen – benutzt habe. Alle Stellen, die wörtlich oder sinngemäß aus Veröffentlichungen entnommen wurden, sind als solche kenntlich gemacht. Ich versichere weiterhin, dass ich die Arbeit vorher nicht in einem anderen Prüfungsverfahren eingereicht habe und die eingereichte schriftliche Fassung der auf dem elektronischen Speichermedium entspricht.

Ggf. streichen: Ich bin damit einverstanden, dass meine Abschlussarbeit in den Bestand der Fachbereichsbibliothek eingestellt wird.

\makeatletter
Hamburg, den {\@date}
\makeatother

\vspace{2cm}
\rule{6cm}{0.25pt}\\
\makeatletter
{\@author} \par
\makeatother


% ================================Deckblatt-Muster==============================
\newpage
\thispagestyle{empty}
% \addcontentsline{toc}{chapter}{Muster des Deckblatts}
\begin{titlepage}% {{{
\includegraphics[width=6.8cm]{../pic/up-uhh-logo-u-2010-u-farbe-u-rgb.pdf}
\begin{center}\Large
	% Universität Hamburg \par
	% Fachbereich Informatik
	\vfill
	Masterarbeit
	\vfill
	\makeatletter
	{\Large\textsf{\textbf{\@title}}\par}
	\makeatother
	\vfill
	vorgelegt von
	\par\bigskip
	\makeatletter
	{\@author} \par
	\makeatother
	geb. am 12. Juni 1991 in Jakarta \par
	Matrikelnummer 6143847 \par
	Studiengang Informatik
	\vfill
	\makeatletter
	eingereicht am {\@date}
	\makeatother
	\vfill
	Betreuer: Dipl.-Inf. Heinz Mustermann \todo{Todos im Text und Fragen an den Betreuer sind in dieser Form dargestellt}\par
	Erstgutachter: Prof. Dr. Chris Biemann \par
	Zweitgutachter: Steffen Remus M.Sc.
\end{center}
\ifoptionfinal{}{
\begin{tikzpicture}[remember picture, overlay]
    \node[draw, red, font=\ttfamily\bfseries\Huge, xshift=50mm, yshift=228mm,
        rotate=340, text centered, text width=8cm, very thick, rounded
        corners=4mm] at (current page.south) {Entwurf vom \today};
\end{tikzpicture}}
\end{titlepage}% }}}

% ================================Literaturliste-Muster==============================
\newpage
\thispagestyle{empty}
\label{sec:literaturliste}
\par\textbf{\textsf{Thema:}} Privacy Enhancing Technologies zum Schutz von Kommunikationsbeziehungen
\par\textbf{\textsf{Bearbeiter:}} Eva Musterfrau, Heinz Mustermann
\par\textbf{\textsf{Datum:}} \today
\bigskip
\par\textbf{\Large\textsf{Literaturliste}}

David Chaum: Untraceable Electronic Mail, Return Addresses, and Digital Pseudonyms. Communications of the ACM 24/2 (1981) 84--88.

David Chaum: The Dining Cryptographers Problem: Unconditional Sender and Recipient Untraceability. Journal of Cryptology 1/1 (1988) 65--75.

David Goldschlag, Michael Reed, Paul Syverson: Onion Routing for Anonymous and Private Internet Connections. Communications of the ACM 42/2 (1999) 39--41.

Andreas Pfitzmann: Diensteintegrierende Kommunikationsnetze mit teilnehmerüberprüfbarem Datenschutz. IFB 234, Springer-Verlag, Berlin 1990.

Wei Wang, Mehul Motani, Vikram Srinivasan: Dependent link padding algorithms for low latency anonymity systems. Proc. 15th ACM conference on Computer and communications security. ACM, 2008, 323--332.


% ================================Todo list==============================
\listoftodos
% \todototoc

\end{document}
