%!TEX encoding = UTF-8 Unicode
\documentclass[
    fontsize=12pt,
    headings=small,
    parskip=half,           % Ersetzt manuelles setzten von parskip/parindent.
    bibliography=totoc,
    numbers=noenddot,       % Entfernt den letzten Punkt der Kapitelnummern.
    open=any,               % Kapitel kann auf jeder Seite beginnen.
%   final                   % Entfernt alle todonotes und den Entwurfstempel.
    ]{scrreprt}

% ===================================Praeambel==================================

% Kodierung, Sprache, Patches {{{
\usepackage[T1]{fontenc}    % Ausgabekodierung; ermoeglicht Akzente und Umlaute
                            %  sowie korrekte Silbentrennung.
\usepackage[utf8]{inputenc} % Erlaub die direkte Eingabe spezieller Zeichen.
                            %  Utf8 muss die Eingabekodierung des Editors sein.
% \usepackage[ngerman]{babel} % Deutsche Sprachanpassungen (z.B. Ueberschriften).
\usepackage{microtype}      % Optimale Randausrichtung und Skalierung.
\usepackage[
    autostyle,
    ]{csquotes}             % Korrekte Anfuehrungszeichen in der Literaturliste.
\usepackage{fixltx2e}       % Patches fuer LaTeX2e.
\usepackage{scrhack}        % Verhindert Warnungen mit aelteren Paketen.
\usepackage[
  newcommands
]{ragged2e}                 % Verbesserte \ragged...Befehle
\PassOptionsToPackage{
  hyphens
}{url}                      % Sorgt für URL-Umbrueche in Fusszeilen u. Literatur
% }}}

% Schriftarten {{{
\usepackage{mathptmx}       % Times; modifies the default serif and math fonts
\usepackage[scaled=.92]{helvet}% modifies the sans serif font
\usepackage{courier}        % modifies the monospace font
% }}}

\usepackage{indentfirst}
\setlength{\parindent}{0.5cm}

% Biblatex {{{
%\usepackage[
%    style=alphabetic,
%    backend=biber,
    %backref=true
%    ]{biblatex}             % Biblatex mit alphabetischem Style und biber.
%\bibliography{bib}      % Dateiname der bib-Datei.
%\DeclareFieldFormat*{title}{
%    \mkbibemph{#1}}         % Make titles italics
% }}}

% Dokument- und Texteinstellungen {{{
\usepackage[
    a4paper,
    margin=2.54cm,
    marginparwidth=2.0cm,
    footskip=1.0cm
    ]{geometry}             % Ersetzt 'a4wide'.
\clubpenalty=10000          % Keine Einzelzeile am Beginn eines Paragraphen
                            %  (Schusterjungen).
\widowpenalty=10000         % Keine Einzelzeile am Ende eines Paragraphen
\displaywidowpenalty=10000  %  (Hurenkinder).
\usepackage{floatrow}       % Zentriert alle Floats.
\usepackage{ifdraft}        % Ermoeglicht \ifoptionfinal{true}{false}
\pagestyle{plain}           % keine Kopfzeilen
% \sloppy                    % großzügige Formatierungsweise
\deffootnote{1em}{1em}{
  \thefootnotemark.\ }      % Verbessert Layout mehrzeiliger Fußnoten

\makeatletter
\AtBeginDocument{%
    \hypersetup{%
        pdftitle = {\@title},
        pdfauthor  = \@author,
    }
}
\makeatother
% }}}

% Weitere Pakete {{{
\usepackage{graphicx}       % Einfuegen von Graphiken.
\usepackage{tabu}           % Einfuegen von Tabellen.
\usepackage{multirow}       % Tabellenzeilen zusammenfassen.
\usepackage{multicol}       % Tabellenspalten zusammenfassen.
\usepackage{booktabs}       % Schönere Tabellen (\toprule\midrule\bottomrule).
\usepackage[nocut]{thmbox}  % Theorembox bspw. fuer Angreifermodell.
\usepackage{amsmath}        % Erweiterte Handhabung mathematischer Formeln.
\usepackage{amssymb}        % Erweiterte mathematische Symbole.
\usepackage{rotating}
\usepackage[
    printonlyused
    ]{acronym}              % Abkuerzungsverzeichnis.
\usepackage[
    colorinlistoftodos,
    textsize=tiny,          % Notizen und TODOs - mit der todonotes.sty von
    \ifoptionfinal{disable}{}%  Benjamin Kellermann ist das Package "changebar"
    ]{todonotes}            %  bereits integriert.
\usepackage[
    breaklinks,
    hidelinks,
    pdfdisplaydoctitle,
    pdfpagemode = {UseOutlines},
    pdfpagelabels,
    ]{hyperref}             % Sprungmarken im PDF. Laed das URL Paket.
    \urlstyle{rm}           % Entfernt die Formattierung von URLs.
%\usepackage{breakurl}
%\def\UrlBreaks{\do\/\do-}
\usepackage{listings}       % Spezielle Umgebung für...
    \lstset{                %  ...Quelltextformatierung.
        language=C,
        breaklines=true,
        breakatwhitespace=true,
        frame=L,
        captionpos=b,
        xleftmargin=6ex,
        tabsize=4,
        numbers=left,
        numberstyle=\ttfamily\footnotesize,
        basicstyle=\ttfamily\footnotesize,
        keywordstyle=\bfseries\color{green!50!black},
        commentstyle=\itshape\color{magenta!90!black},
        identifierstyle=\ttfamily,
        stringstyle=\color{orange!90!black},
        showstringspaces=false,
        }
\usepackage{filecontents}   % Direktes Einfuegen von Dateiinhalt. Wird hier fuer
                            %  die Verwendung einer .bib-Datei in dieser .tex-
                            %  Datei benoetig.
% }}}

% ===================================Dokument===================================

\title{Master Thesis}
\author{Alvin Rindra Fazrie}
% \date{01.01.2015} % falls ein bestimmter Tag eingesetzt werden soll, einfach
                    %  diese Zeile aktivieren

\begin{document}

\begin{titlepage}
\includegraphics[width=6.8cm]{../pic/up-uhh-logo-u-2010-u-farbe-u-rgb.pdf}
  \setcounter{page}{-1}

	% Titelseite
	%\begin{figure}[h]
	%	\begin{minipage}[b]{62mm}
	%		\includegraphics[width=62mm]{../pic/up-uhh-logo-u-2010-u-farbe-u-rgb.pdf}
	%	\end{minipage}
	%	\hspace{4cm}
		%\begin{minipage}[b]{59mm}
		%	\includegraphics[width=59mm]{images/minlogo}
		%\end{minipage}
	%\end{figure}

	\vfill
\begin{center} 
		\noindent { \huge
			Master thesis \\
		}
		\vspace{14mm}
		% Titel
		\noindent \textbf{\huge
		  Autolinks: Adaptive Hypergraph-based Information Management of Semantic Triples 
		}
		\vspace{60mm}	
	\end{center}
	
	\vfill
	
	\noindent{\textbf{Alvin Rindra Fazrie}} \\
	\noindent \rule{\textwidth}{0.4mm} 
	\noindent{\textrm{4fazrie@informatik.uni-hamburg.de}} \\
	\noindent{\textrm{Intelligent Adaptive Systems Master program}} \\
	\noindent{\textrm{Matr.-Nr. 6641834}} \\
	\begin{tabbing}
	\hspace{8em} \=  \kill
	First Supervisor: \> Prof. Dr. Chris Biemann \\
	Second Supervisor: \> Steffen Remus MSc. \\
	\end{tabbing}
\end{titlepage}


\chapter*{Abstract}

 Autolinks 'automatic proactive researching’ is a tool that provides a quick researching platform based on a text or a sentence by visualizing the results together with their semantic relations. In this internet era, people could get information easily with search engines. They will give us a ton of hyperlinks clustered by multiple pages by entering a single query to the input, then we could select a specific link we think the most relevant. The process of learning takes a time sometimes. After the chosen web page rendered, we need to read through a page to get a specific information related to the query given and sometimes we still have to deal with a number of hyperlinks to get further information. Even worse, most of the website nowadays exploit the curiosity gap of the reader, providing just enough information and not enough to satisfy the reader’s curiosity, without clicking through another linked content. This clickbait phenomenon becomes so normal today and it makes our time to study longer. 
 
    	Autolinks optimizes these concerns and is intended to make the learning process faster and more efficient. Instead of reading papers, websites, and other resources to understand a specific term, this machine will do it for us. From a text or a sentence given by the user, it will read and learn from multiple resources and digests the core related information by visualizing the information in the most convenient way. The information is visualized by a force-directed graph, a graph which contains nodes for the information and edges for the semantic relation so that it will ease the reader to understand how pieces of information correlate each other.
    	
    	Autolinks is built with machine learning paradigm. Natural Language Processing (NLP) takes a responsibility to understand a given text and to comprehend which information from the sources have a relation to the given text and correlate each other. The reader could evaluate the results given and Autolinks will learn and correct the mistakes so that it could improve the precision and confidence in the next iteration. Bundled with this capability, Autolinks accelerates the process of researching and understanding during the study.  
    	
	With respect to the background and the purpose of Autolinks, we address some research questions in this master thesis, including the following: how can a user interface be devised that is non-intrusive, i.e. helping users solve their information needs faster instead of impeding them?; which semantic services, realized with NLP technologies, are the most useful?; how can we measure success, i.e. showing that Autolinks really live up to its premise?

\chapter*{Acknowledgement}

Here comes the acknowledgement...


\tableofcontents

\chapter{Introduction}

Interaction between Artificial Intelligence and humans has been growing to be a vital part in daily life in the past decade. Natural Language Processing has become one of the fundamental topics for scientists in the field of Data Science. In this thesis, we introduce the basic concepts of Language Technology and expand on the idea of visual interaction to support the learning / research process. To this end, we develop autolinks, cytoscape \cite{doi:10.1093/bioinformatics/btv557}.

- Graph Theory, cytoscape.js \cite{doi:10.1093/bioinformatics/btv557}

- Hypergraph Papers: 

An Algorithm for drawing compound graphs \cite{10.1007/3-540-46648-7_20}

A Layout algorithm for undirected compund graphs \cite{DOGRUSOZ2009980}

Graph Visualization Techniques for Web Clustering Engines \cite{4069238}

\section{Motivation}

Visual Analytics \cite{1573625}
Visual Analytics methode \cite{1333626}
Visual Analytics Process \cite{CCCW2009}

Taxonomy of Information Visualization \cite{545307}


An algorithm that learn what is in name \cite{Bikel1999}


\section{Research question}

Here comes the the Research Questions...

\begin{enumerate}
	\item How can a user interface be devised that is non-intrusive, i.e. helping users solve their information needs faster instead of impeding them?
	\item Which semantic services, realized with NLP technologies, are the most useful?
	\item How can we measure success, i.e. showing that Autolinks really live up to its premise?,
\end{enumerate}


\section{Contributions}

Here comes the contributions..


\chapter{Related Work}

Software for Social Network Analysis: \cite{M_Huisman}


\section{Information Management tools}

Here comes information management tools...


\section{Text annotation tools}

Here comes text annotation tools...



\chapter{Background Study}

Here comes Background Study...


\section{Language Technology}

Here comes contributions...


\section{Machine learning}

Here comes Machine Learning...


\section{Data Visualization}

Here comes Data Visualization...

\section{Hypergraph}

Here comes Hypergraph...

\section{Web technologies}

Here comes Web technologies...



\chapter{System Overview}

Here comes System Overview...

\section{Autolinks Introduction}

Here comes Autolinks Introduction...


\section{Components in Autolinks}

Here comes components in Autolinks...


\chapter{Data Extraction}

Here comes Data Extraction...

\section{Broker}

Here comes broker...

\section{Wiki Service}

Here comes Wikiservice...


\chapter{Information Management Visualization}

Here comes Information Management Viz...

\section{Concept and Visualization}

Here comes Concept and Visualization...

\section{Compound Nodes / Parent}

Here comes Compound Nodes / Parent...



\section{System Overview}

Here comes the System Overview...

\section{Data Extraction}

Here comes the Data Extraction...

Example of lists:

\begin{enumerate}
	\item Fachbücher, Standards,
	\item Wiss. Zeitschriftenartikel, Survey-Artikel,
	\item Konferenzbeiträge,
	\item Technical Reports, graue Literatur,
	\item Online-Material, Arbeitspapiere, Firmenmaterial, Ausarbeitungen.
\end{enumerate}

Im Internet können zur Feststellung der Qualität und Recherche von Publikationen

\begin{itemize}
	\item Google Scholar (\url{http://scholar.google.com}),
	\item Microsoft Academic Search (\href{http://academic.research.microsoft.com/?SearchDomain=2&SubDomain=2&entitytype=2}{http://academic.research.microsoft.com}) $\to$ computer science $\to$ security \& privacy,
	\item Computer Science Bibliography (\url{http://dblp.uni-trier.de/}) und die
	\item Scientific Literature Digital Library (\url{http://citeseer.nj.nec.com/})
\end{itemize}



\chapter{Evaluation}

Here comes the evaluation...

\section{Case Study}

Here comes the case study...

\section{User Experiment}

Here comes the user experiment...

\section{Evaluation details}

Here comes the evaluation details...

\begin{lstlisting}[float,caption={Example of algorithm},label={lst:ggt}]
int getGGTOf(int a, int b) {
    // requires ((a > 0) && (b > 0)); ensures return > 0;
    int h;
    while (b != 0) {
        h = b;
        b = a % b; // % is the modulo operator. This line is long enough to show how line breaks in lstlisting are handled.
        a = h;
    }
    return a;
}
\end{lstlisting}


\chapter{Future Work}

Here comes the future work...


\chapter{Conclusion}

Here comes the conclusion...

% =============================Literaturverzeichnis=============================
%\begin{raggedright}         % Schaltet Blocksatz ab, erzeugt ein stimmigeres
                            %  Schriftbild im Literaturverzeichnis.
%  \printbibliography        % Falls Biblatex verwendet wird.
%  \label{sec:literaturverzeichnis}
%\end{raggedright}

%%%%%%%%%%%%%%%%%%%%%%%%%%%%%%%%%%%%%%
% hier werden - zum Ende des Textes - die bibliographischen Referenzen
% eingebunden
%
% Insbesondere stehen die eigentlichen Informationen in der Datei
% ``bib.bib''
%
\newpage
\bibliographystyle{plain}
\addcontentsline{toc}{section}{Bibliography}% Add to the TOC
\bibliography{bib}

\end{document}
